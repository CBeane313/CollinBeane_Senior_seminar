%% The first command in your LaTeX source must be the \documentclass command.

\documentclass[sigplan,screen,nonacm]{acmart}
\usepackage{color}
\setlength {\marginparwidth }{2cm}
\usepackage[colorinlistoftodos]{todonotes}
\usepackage[
    type={CC},
    modifier={by-nc-sa},
    version={4.0},
]{doclicense}

%% \BibTeX command to typeset BibTeX logo in the docs
\AtBeginDocument{%
  \providecommand\BibTeX{{%
    \normalfont B\kern-0.5em{\scshape i\kern-0.25em b}\kern-0.8em\TeX}}}

%% end of the preamble, start of the body of the document source.

\begin{document}

%%
%% The "title" command has an optional parameter,
%% allowing the author to define a "short title" to be used in page headers.
\title{Using Internet of Things for Wildlife Tracking}

%% "authornote" and "authornotemark" commands
%% used to denote shared contribution to the research.
\author{Collin R. Beane}
\email{beane039@morris.umn.edu}
\affiliation{%
  \institution{Division of Science and Mathematics 
	\\
        University of Minnesota, Morris
	}
  \city{Morris}
  \state{Minnesota}
  \country{USA}
  \postcode{56267}
}

%%
%% The abstract is a short summary of the work to be presented in the
%% article.
\begin{abstract}
    This paper provides a comprehensive examination of the utilization of 
    Internet of Things (IoT) devices in wildlife management and tracking, 
    their evolutionary trajectory, and practical implementation in data 
    acquisition. Central to the discussion are key components of IoT networks, 
    including Sigfox, Wi-Fi-enabled devices, and IoT-based wireless sensor 
    networks, each analyzed for their role and efficacy. Communication 
    modalities within IoT frameworks, coupled with an evaluation of protocol 
    performance are evaluated.

    Furthermore, this seminar also addresses challenges inherent in wildlife 
    data collection methodologies, such as memory constraints, battery life, 
    transmission range and rate, and security vulnerabilities within IoT 
    ecosystems. By delving into potential solutions and technological 
    advancements, this paper aims to contribute to the refinement of wildlife 
    monitoring practices, fostering a more robust and effective approach to 
    conservation efforts.
    \todo[inline, color=green]{This is a preliminary abstract, I mainly added it just so I had something.}
\end{abstract}

\doclicenseThis

%%
%% Keywords. The author(s) should pick words that accurately describe
%% the work being presented. Separate the keywords with commas.
\keywords{IoT, networking, Wi-Fi, data transmission, data collection, animal trackers, Sigfox, WildFi}


%%
%% This command processes the author and affiliation and title
%% information and builds the first part of the formatted document.
\maketitle

\section{Introduction}
\label{sec:introduction}

\section{Background}
\label{sec:Background}
\todo[inline, color=green]{I'm not sure whether or not this is the right 
place for background information on what biologging is and its history or 
if that sort of stuff belongs in the introduction, I will include it for now}

Comprehending the foundational technology behind the Internet of Things (IoT) 
is paramount in grasping its applications in wildlife tracking. This section 
aims to furnish a concise overview of biologging, the IoT, and their 
intersection in wildlife tracking. Additionally, it will explore current and past 
technologies employed in biologging, shedding light on their operational 
mechanisms and comparative advantages. By delving into the workings of 
traditional wildlife tracking technologies, we can evaluate their merits and 
demerits, thereby establishing a framework for evaluating the suitability of 
IoT solutions for wildlife tracking.

\subsection{What is Biologging}
\label{subsec:What is Biologging}

\subsection{What is the Internet of Things}
\label{subsec:What is the Internet of Things}

\section{Components}
\label{sec:Components}

\section{Data Transmission}
\label{sec:Data Transmission}

\section{Networking Protocols}
\label{sec:Networking Protocols}

\section{Challenges to Overcome}
\label{sec:Challenges to Overcome}

%% Elena: examples like this are actually helpful, but I can't locate theit bibliography 
%% file (I think they are using a databse), so providing examples will have to wait.

%  Some examples.  A paginated journal article \cite{Abril07}, an
%  enumerated journal article \cite{Cohen07}, a reference to an entire
%  issue \cite{JCohen96}, a monograph (whole book) \cite{Kosiur01}, a
%  monograph/whole book in a series (see 2a in spec. document)
%  \cite{Harel79}, a divisible-book such as an anthology or compilation
%  \cite{Editor00} followed by the same example, however we only output
%  the series if the volume number is given \cite{Editor00a} (so
%  Editor00a's series should NOT be present since it has no vol. no.),
%  a chapter in a divisible book \cite{Spector90}, a chapter in a
%  divisible book in a series \cite{Douglass98}, a multi-volume work as
%  book \cite{Knuth97}, a couple of articles in a proceedings (of a
%  conference, symposium, workshop for example) (paginated proceedings
%  article) \cite{Andler79, Hagerup1993}, a proceedings article with
%  all possible elements \cite{Smith10}, an example of an enumerated
%  proceedings article \cite{VanGundy07}, an informally published work
%  \cite{Harel78}, a couple of preprints \cite{Bornmann2019,
%    AnzarootPBM14}, a doctoral dissertation \cite{Clarkson85}, a
%  master's thesis: \cite{anisi03}, an online document / world wide web
%  resource \cite{Thornburg01, Ablamowicz07, Poker06}, a video game
%  (Case 1) \cite{Obama08} and (Case 2) \cite{Novak03} and \cite{Lee05}
%  and (Case 3) a patent \cite{JoeScientist001}, work accepted for
%  publication \cite{rous08}, 'YYYYb'-test for prolific author
%  \cite{SaeediMEJ10} and \cite{SaeediJETC10}. Other cites might
%  contain 'duplicate' DOI and URLs (some SIAM articles)
%  \cite{Kirschmer:2010:AEI:1958016.1958018}. Boris / Barbara Beeton:
%  multi-volume works as books \cite{MR781536} and \cite{MR781537}. A
%  couple of citations with DOIs:
%  \cite{2004:ITE:1009386.1010128,Kirschmer:2010:AEI:1958016.1958018}. Online
%  citations: \cite{TUGInstmem, Thornburg01, CTANacmart}. Artifacts:
%  \cite{R} and \cite{UMassCitations}.


%%
%% The acknowledgments section is defined using the "acks" environment
%% (and NOT an unnumbered section). This ensures the proper
%% identification of the section in the article metadata, and the
%% consistent spelling of the heading.
\begin{acks}
This is where you thank those who helped you better understand the material 
and gave you helpful feedback on the paper, usually including your adviser. 
This is not a place to thank your family, your significant other or your best friend, 
or anyone else  for moral support or yummy cookies. 
\end{acks}

%%
%% The next two lines define the bibliography style to be used, and
%% the bibliography file.
\bibliographystyle{ACM-Reference-Format}
\bibliography{sample_paper}


\end{document}
\endinput
%%
%% End of file `sample-sigplan.tex'.
