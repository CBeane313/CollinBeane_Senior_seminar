%% The first command in your LaTeX source must be the \documentclass command.

\documentclass[sigplan,screen,nonacm]{acmart}
\usepackage{color}
\setlength {\marginparwidth }{2cm}
\usepackage[colorinlistoftodos]{todonotes}
\usepackage[
    type={CC},
    modifier={by-nc-sa},
    version={4.0},
]{doclicense}

%% \BibTeX command to typeset BibTeX logo in the docs
\AtBeginDocument{%
  \providecommand\BibTeX{{%
    \normalfont B\kern-0.5em{\scshape i\kern-0.25em b}\kern-0.8em\TeX}}}

%% end of the preamble, start of the body of the document source.

\begin{document}

%%
%% The "title" command has an optional parameter,
%% allowing the author to define a "short title" to be used in page headers.
\title{Using Internet of Things for Wildlife Tracking}

%% "authornote" and "authornotemark" commands
%% used to denote shared contribution to the research.
\author{Collin R. Beane}
\email{beane039@morris.umn.edu}
\affiliation{%
  \institution{Division of Science and Mathematics 
	\\
        University of Minnesota, Morris
	}
  \city{Morris}
  \state{Minnesota}
  \country{USA}
  \postcode{56267}
}

%%
%% The abstract is a short summary of the work to be presented in the
%% article.
\begin{abstract}
    This paper provides a comprehensive examination of the utilization of 
    Internet of Things (IoT) devices in wildlife management and tracking, 
    their evolutionary trajectory, and practical implementation in data 
    acquisition. Central to the discussion are key components of IoT networks, 
    including Sigfox, Wi-Fi-enabled devices, and IoT-based wireless sensor 
    networks, each analyzed for their role and efficacy. Communication 
    modalities within IoT frameworks, coupled with an evaluation of protocol 
    performance are evaluated.

    Furthermore, this seminar also addresses challenges inherent in wildlife 
    data collection methodologies, such as memory constraints, battery life, 
    transmission range and rate, and security vulnerabilities within IoT 
    ecosystems. By delving into potential solutions and technological 
    advancements, this paper aims to contribute to the refinement of wildlife 
    monitoring practices, fostering a more robust and effective approach to 
    conservation efforts.
    \todo[inline, color=orange]{This is a preliminary abstract, I mainly added it just so I had something.}
\end{abstract}

\doclicenseThis

\keywords{IoT, networking, Wi-Fi, data transmission, data collection, 
animal trackers, Sigfox, WildFi, Biologging, ecology}


%%
%% This command processes the author and affiliation and title
%% information and builds the first part of the formatted document.
\maketitle

\section{Introduction}
\label{sec:introduction}

\section{Background}
\label{sec:Background}
\todo[inline, color=orange]{I'm not sure whether or not this is the right 
place for background information on what biologging is and its history or 
if that sort of stuff belongs in the introduction, I will include it for now}

Comprehending the foundational technology behind the Internet of Things (IoT) 
is paramount in grasping its applications in wildlife tracking. This section 
aims to furnish a concise overview of biologging, the IoT, and their 
intersection in wildlife tracking. Additionally, it will explore current and past 
technologies employed in biologging, shedding light on their operational 
mechanisms and comparative advantages. By delving into the workings of 
traditional wildlife tracking technologies, we can evaluate their merits and 
demerits, thereby establishing a framework for evaluating the suitability of 
IoT solutions for wildlife tracking.

\subsection{What is Biologging?}
\label{subsec:What is Biologging}

Biologging is a concept that gained popularity in the early 2000's and has continued 
to play a pivotal role in understanding animal behavior and ecology. Biologging can be 
defined as 
\begin{quote}
  The investigation of phenomena in or around free-ranging organisms that are beyond 
  the boundary of our visibility or experience. \cite{boyd2004bio}
\end{quote}
It is a method of tracking animals in the wild using electronic devices that are 
attached to the animal. These devices can be used to track the animal's 
movements, monitor its behavior, and collect data on its environment. Biologging 
emerged as a powerful tool in ecology the same way genomics did for the study 
of cellular and organ function. The primary difference being that biologging provides 
insights into the behavior and functions of various organisms in environments that 
can be hostile or difficult to reach for the observer \cite{boyd2004bio}. The 
ability to track animals in their natural environment has provided researchers 
with a wealth of data that was previously unattainable. This data has been used 
to study animal behavior, migration patterns, and the effects of climate change 
on various species\cite{10.3389/fevo.2018.00092}. The data collected from biologging devices has also been 
used to inform conservation efforts and to help protect endangered species \cite{cooke2008biotelemetry}. 
It is important to understand that biologging is simply the collection of data
from animals in the wild, and it is then up to scientists or conservationists to use 
the data to answer questions about the animals or to inform conservation efforts. 

\subsection{What is the Internet of Things?}
\label{subsec:What is the Internet of Things}

The Internet of Things (IoT) represents a transformative shift in the 
realm of technology, encompassing a vast array of physical objects empowered 
with sensors and software to interact autonomously. These objects collect and 
exchange data through network connectivity. In essence, IoT devices, ranging 
from commonplace gadgets to sophisticated systems, have the capability to 
interface with the internet or communicate wirelessly, thereby facilitating 
seamless integration into various facets of daily life. The IoT has been 
applied to a wide range of fields, including healthcare, agriculture, 
manufacturing, and most important to this paper, wildlife monitoring. The 
fundamental operational mechanism of the IoT hinges upon established protocols 
such as Internet Protocol (IP) and Transmission Control Protocol (TCP), which 
serve as the foundational infrastructure for enabling connectivity among 
sensors, devices, and networks. Data generated by IoT devices are processed 
and transmitted across a myriad of wired and wireless communication channels, 
encompassing Ethernet, Wi-Fi, Bluetooth, cellular networks like 5G and LTE, 
radio frequency identification (RFID), and near field communication (NFC)\cite{greengard2021internet}.
\todo[inline, color=orange]{This source is a book, but there is a britanica page where I 
found the info I used, same author and everything, but the britanica page is not the book. 
is citing the book but using the britanica info ok?}
There are two primary types of IoT devices, digital-first and physical-first. 
Digital-first devices are designed and built with connectivity in mind, and 
encompass a diverse range of machines and gadgets capable of generating and 
transmitting data. Conversely, physical-first devices incorporate microchips 
or sensors retrofitted with communication capabilities, thereby imbuing 
conventional objects with newfound functionality and traceability \cite{greengard2021internet}.

\subsection{What are the Other Biologging Methods?}
\label{subsec:What are the Other Biologging Methods?}

\section{Components}
\label{sec:Components}

\section{Data Transmission}
\label{sec:Data Transmission}

\section{Networking Protocols}
\label{sec:Networking Protocols}

\section{Challenges to Overcome}
\label{sec:Challenges to Overcome}

%% Elena: examples like this are actually helpful, but I can't locate theit bibliography 
%% file (I think they are using a databse), so providing examples will have to wait.

%  Some examples.  A paginated journal article \cite{Abril07}, an
%  enumerated journal article \cite{Cohen07}, a reference to an entire
%  issue \cite{JCohen96}, a monograph (whole book) \cite{Kosiur01}, a
%  monograph/whole book in a series (see 2a in spec. document)
%  \cite{Harel79}, a divisible-book such as an anthology or compilation
%  \cite{Editor00} followed by the same example, however we only output
%  the series if the volume number is given \cite{Editor00a} (so
%  Editor00a's series should NOT be present since it has no vol. no.),
%  a chapter in a divisible book \cite{Spector90}, a chapter in a
%  divisible book in a series \cite{Douglass98}, a multi-volume work as
%  book \cite{Knuth97}, a couple of articles in a proceedings (of a
%  conference, symposium, workshop for example) (paginated proceedings
%  article) \cite{Andler79, Hagerup1993}, a proceedings article with
%  all possible elements \cite{Smith10}, an example of an enumerated
%  proceedings article \cite{VanGundy07}, an informally published work
%  \cite{Harel78}, a couple of preprints \cite{Bornmann2019,
%    AnzarootPBM14}, a doctoral dissertation \cite{Clarkson85}, a
%  master's thesis: \cite{anisi03}, an online document / world wide web
%  resource \cite{Thornburg01, Ablamowicz07, Poker06}, a video game
%  (Case 1) \cite{Obama08} and (Case 2) \cite{Novak03} and \cite{Lee05}
%  and (Case 3) a patent \cite{JoeScientist001}, work accepted for
%  publication \cite{rous08}, 'YYYYb'-test for prolific author
%  \cite{SaeediMEJ10} and \cite{SaeediJETC10}. Other cites might
%  contain 'duplicate' DOI and URLs (some SIAM articles)
%  \cite{Kirschmer:2010:AEI:1958016.1958018}. Boris / Barbara Beeton:
%  multi-volume works as books \cite{MR781536} and \cite{MR781537}. A
%  couple of citations with DOIs:
%  \cite{2004:ITE:1009386.1010128,Kirschmer:2010:AEI:1958016.1958018}. Online
%  citations: \cite{TUGInstmem, Thornburg01, CTANacmart}. Artifacts:
%  \cite{R} and \cite{UMassCitations}.


%%
%% The acknowledgments section is defined using the "acks" environment
%% (and NOT an unnumbered section). This ensures the proper
%% identification of the section in the article metadata, and the
%% consistent spelling of the heading.
\begin{acks}
This is where you thank those who helped you better understand the material 
and gave you helpful feedback on the paper, usually including your adviser. 
This is not a place to thank your family, your significant other or your best friend, 
or anyone else  for moral support or yummy cookies. 
\end{acks}

%%
%% The next two lines define the bibliography style to be used, and
%% the bibliography file.
\bibliographystyle{ACM-Reference-Format}
\bibliography{sample_paper}


\end{document}
\endinput
%%
%% End of file `sample-sigplan.tex'.
